%---------------------------------------------------------------------------%
%->> Backmatter
%---------------------------------------------------------------------------%
\chapter[致谢]{致\quad 谢}\chaptermark{致\quad 谢}% syntax: \chapter[目录]{标题}\chaptermark{页眉}
\thispagestyle{noheaderstyle}% 如果需要移除当前页的页眉
%\pagestyle{noheaderstyle}% 如果需要移除整章的页眉

在论文接近尾声的时候,仿佛看到了我的博士生活也接近尾声了。
在物理所生活的五年,是我学生生涯的最后阶段,也是我收获最丰富的五年。这里有太多太多的美好值得回忆,我希望在这里向每一位曾经帮助过我的人说一声谢谢。

首先要感谢的就是我的导师翁红明老师。翁老师不仅是我科研工作的领路人,也在我科研生活的方方面面影响着我。%第一次来北京找翁老师的画面仍然历历在目:翁老师先是询问了我的科研兴趣,然后认真地向我介绍了课题组的情况,并仔细的告诉我如何在网上找到老师的文献,最后还送给我两本杂志,并嘱咐我认真阅读文献。
刚进组时,每次和老师讨论,或是在电梯里碰面,翁老师都不厌其烦地告诉我一定要多读文献。虽然我的基础不好,但是翁老师总是会循循善诱地给我讲解一些很基础的问题,严格要求我对每一个细节都要掌握清楚,不能模棱两可,指导我一步步从对科研一无所知到可以独当一面。翁老师还经常指导我和一些实验组合作,在这个过程中,我发现翁老师不光有丰富的理论知识,对实验的细节也驾轻就熟。老师告诉我这些知识就是要靠不断的积累、学习。
在这些合作中,我不仅向实验组的老师和同学们学到很多知识,也让我意识到不光要会计算,还要能理解实验的可行性,这对于我今后寻找有价值的课题非常重要。在组里的这几年,翁老师还为我提供了很多学习的机会,包括组里举办的一些workshop,其他单位组织的关于拓扑理论和第一性原理相关的会议。每次会议我都能学习到很多知识,还能有机会见识自己所在的领域里的各位前辈们的科研品味、态度和精神,有时还有机会能结交到一些很优秀的同龄人,这些都是我今后科研道路上的宝贵财富。

我还要感谢‪Oleg Yazyev‬老师和吴泉生师兄、张胜男师姐对我科研工作和生活初期的耐心指导。Oleg老师是一个非常耐心的老师,经常对当时几乎什么都不懂的我给予鼓励。他看待问题总是可以做到一针见血,每次组会上,当我刚刚把自己遇到的困难说出来,他就能立马抓住问题的本质,告诉我可以在哪些方面寻求突破。吴师兄能力非常强,不仅有扎实的理论功底上,超强的编程能力,还有非常强的学习能力,独特的科研品味。我在第一性原理计算方面的相关知识几乎是吴师兄手把手教的,师兄总是很有耐心的一边指导我,一边鼓励我,还“骗”我他读研时也像我一样什么都不懂。在师兄的“欺骗”下,我一直自我感觉良好,虽然遇到一些困难,但也能够乐观面对,这使得我在后知后觉中一步步走上科研的正轨。张胜男师姐在生活方面给了我非常非常多的帮助,在科研上我遇到困难时也经常鼓励我,在此表示真心的感谢。

我还想感谢方辰老师。和方老师的合作并不多,但是方辰却一直是我的精神偶像。听过方老师的很多报告,也拜读过方老师的很多工作,可以说我在拓扑理论方面的大部分知识是从方老师的工作中学习到的。我的LaSbTe这个工作也是完全依赖于方辰老师层构造的方法。那一次方老师一遍遍认真地帮我修改论文,还告诫我论文里一定不能有低级错误。后来又有幸经常参加方老师组内聚会,聆听老师对科研的认识和见解。在我博后申请方面,方老师也帮了我很大的忙,不仅帮我写推荐信,还认真地教我如何正确使用一些语法。这次的三月会议,也有幸聆听了方老师对我讲解的工作提供的指导意见。唯一遗憾的是我一直因为比较怂,有问题却不敢向方老师请教,总怕被老师嫌弃而丧失了很多次和老师学习的机会。虽然在紧张的科研工作和生活中与方老师的交集不多,但深深被方老师的为人所折服。希望今后也能像方老师一样在科研和生活两方面做好自己。

我还想感谢王志俊师兄对我科研工作的指导。王师兄工作非常努力,认真。师兄在科研工作中事无巨细,让我学到了很多。师兄的想法很多,给了我很多材料去探索,我能够顺利毕业也离不开王师兄的指导。

我还想感谢每一位合作者,感谢方忠老师,物理所石友国老师,周兴江老师,刘国东老师,钱天老师,王建涛老师,靳常青老师,刘淼老师,雒建林老师,毛寒青老师,遵义师范大学的谭志云老师,人大的雷和畅老师,高嘉成师弟,聂思敏师兄,张坦师兄,崔志海师兄,张薇师姐,宋志达师兄,蒋毅师弟,杨萌师姐,伊长江师兄,孔令元师兄,王阳师姐,吴德胜同学,卜坤师弟,赵建发师兄,樊文辉同学,李勇师弟,刘清波同学等,与你们的合作让我受益匪浅!

感谢理论室的王磊老师,孟子杨老师,万源老师,周毅老师,徐力方老师,齐建为老师,于艳梅老师,刘伍明老师,李晶晶老师,边智聪老师,王静静老师等各位老师在科研工作和生活中的帮助!感谢组里徐刚,赵建洲,程秋波,邵德喜,杨健,
%郭照\hbox{\lower-0.7ex\hbox{\scalebox{1}[0.9]{艹}}\lower.1ex\hbox{\kern-1em \scalebox{1}[0.7]{凡}}}
许秋楠,顾越强,张田田,周丽琴,皮涵琦,彭炳睿,李楚豪,李烁辉,刘毓智,潘高培,岳长明,徐远锋,任宏斌,彭士宇,孙松,李哲,陈闯,廖元达,任杰,刘子宏,孙光宇,邵岳林,姬学聪,张中义,张帅,王薇,王瑶,张锴,贾玉锦,吴东宇,邓俊泽,朱天念,高恒,梁英宗,王珊珊,钱晨等师兄弟姐妹们,与你们学习生活,让我成长到了很多!诚然,成年人的世界里没有永远的陪伴,纵有千般不舍,终究还是要前行。愿T03组永远和谐温暖,朝气蓬勃,加油!

感谢我的爱人胡志华这些年来对我科研工作一如既往的支持,对我的生活无微不至的关怀。最后我想感谢我的父母在家境拮据的情况下,一直供我读书,支持我读博,甚至出国深造。仅以此论文献给我亲爱的父母亲!

\cleardoublepage[plain]% 让文档总是结束于偶数页,可根据需要设定页眉页脚样式,如 [noheaderstyle]
%---------------------------------------------------------------------------%


\chapter{作者简历及攻读学位期间发表的学术论文与研究成果}

%\textbf{本科生无需此部分}。

\section*{作者简历}

钱玉婷,女,河北省阳原县人,1993年8月出生,中国科学院物理研究所博士
研究生。
\section*{教育经历}
2012年9月 - 2016年6月,就读于河北师范大学物理系,物理学专业,获得学士学位

2016年9月 - 2021年6月,就读于中国科学院物理研究所,理论物理专业硕博连读,获得博士学位

\section*{参加的研究项目及获奖情况:}

2018年-2019年 中国科学院物理研究所所长奖学金表彰奖 

2019年-2020年 中国科学院物理研究所所长奖学金表彰奖 

2019年-2020年 中国科学院大学三好学生 

\clearpage
\section*{已发表(或正式接受)的学术论文:}

{
\setlist[enumerate]{}% restore default behavior
\begin{enumerate}[nosep]
    \item Gao J, \textbf{Qian Y}, Nie S, et al. High-throughput screening for weyl semimetals with S$_4$ symmetry [J]. Science Bulletin, 2021, 66(7): 667-675.
    \item Wang Y$^*$, \textbf{Qian Y}$^*$, Yang M$^*$, et al. Spectroscopic evidence for the realization of a genuine topological nodal-line semimetal in LaSbTe [J]. Phys. Rev. B, 2021, 103: 125131.
    \item Zhao J, Gao J, Li W, \textbf{Qian Y} et al. A combinatory ferroelectric compound bridging simple ABO$_3$ and A-site-ordered quadruple perovskite [J]. Nature communications, 2021, 12(1): 1-9.
    \item Bu K$^*$, \textbf{Qian Y}$^*$, Wang J, et al. Hybrid nodal chain in an orthorhombic graphene network [J]. Physical Review B, 2021, 103(8): L081108.

    \item \textbf{Qian Y}$^*$, Tan Z$^*$, Zhang T, et al. Layer construction of topological crystalline insulator LaSbTe [J]. Science China: Physics, Mechanics and Astronomy, 2020, 63(10).
    \item \textbf{Qian Y}$^*$, Gao J$^*$, Song Z, et al. Weyl semimetals with S$_4$ symmetry [J]. Phys. Rev. B, 2020, 101: 155143.
    \item Cui Z, \textbf{Qian Y}, Zhang W, et al. Type-II Dirac Semimetal State in a Superconductor Tantalum Carbide [J]. Chinese Physics Letters, 2020, 37(8).
    \item Tian S, Gao S, Nie S, et al. Magnetic topological insulator MnBi$_6$Te$_{10}$ with a zero-field ferromagnetic state and gapped Dirac surface states [J]. Phys. Rev. B, 2020, 102: 035144.
    \item Wu D, \textbf{Qian Y}, Liu Z, et al. Single crystal growth, structural and transport properties of bad metal RhSb$_2$ [J]. Chinese Physics B, 2020, 29(3): 037101.
    \item Yang M$^*$, \textbf{Qian Y}$^*$, Yan D, et al. Magnetic and electronic properties of a topological nodal line semimetal candidate: HoSbTe [J]. Phys. Rev. Materials, 2020, 4: 094203.
    \item Yue S$^*$, \textbf{Qian Y}$^*$, Yang M, et al. Topological electronic structure in the antiferromagnet HoSbTe [J]. Phys. Rev. B, 2020, 102: 155109.
    \item An L, Zhu X, \textbf{Qian Y}, et al. Signature of Dirac semimetal states in gray arsenic studied by de Haas–van Alphen and Shubnikov–de Haas quantum oscillations [J]. Phys. Rev. B, 2020, 101: 205109.
    \item Nie S$^*$, \textbf{Qian Y}$^*$, Gao J, et al. The application of topological quantum chemistry in electrides [J]. Phys. Rev. B, 2021,  103: 205133.
    %2020: arXiv:2012.02203.

    \item \textbf{Qian Y}, Nie S, Yi C, et al. Topological electronic states in HfRuP family superconductors [J]. npj Computational Materials, 2019: 5, 121.
    \item Wang J, \textbf{Qian Y}, Weng H, et al. Three-dimensional crystalline modification of graphene in all-sp$^2$ hexagonal lattices with or without topological nodal lines [J]. The Journal of Physical Chemistry Letters, 2019, 10(10): 2515-2521.
    \item Liu Q, \textbf{Qian Y}, Fu H, et al. Symmetry-Enforced Weyl Phonons [J]. npj Computational Materials, 2019: 6, 95.
    \item Xu N, \textbf{Qian Y}, Wu Q, et al. Trivial topological phase of CaAgP and the topological nodal-line transition in CaAg(P$_{1−𝑥}$As$_𝑥$) [J]. Phys. Rev. B, 2018, 97: 161111.
    



\end{enumerate}
}

% \chapter[致谢]{致\quad 谢}\chaptermark{致\quad 谢}% syntax: \chapter[目录]{标题}\chaptermark{页眉}
% \thispagestyle{noheaderstyle}% 如果需要移除当前页的页眉
% %\pagestyle{noheaderstyle}% 如果需要移除整章的页眉

% 在论文接近尾声的时候,仿佛看到了我的博士生活也接近尾声了。
% 在物理所生活的五年,是我学生生涯的最后阶段,也是我收获最丰富的五年。这里有太多太多的美好值得回忆,我希望在这里向每一位曾经帮助过我的人说一声谢谢。

% 首先要感谢的就是我的导师翁红明老师。翁老师不仅是我科研工作的领路人,也在我科研生活的方方面面影响着我。%第一次来北京找翁老师的画面仍然历历在目:翁老师先是询问了我的科研兴趣,然后认真地向我介绍了课题组的情况,并仔细的告诉我如何在网上找到老师的文献,最后还送给我两本杂志,并嘱咐我认真阅读文献。
% 刚进组时,每次和老师讨论,或是在电梯里碰面,翁老师都不厌其烦地告诉我一定要多读文献。虽然我的基础不好,但是翁老师总是会循循善诱地给我讲解一些很基础的问题,严格要求我对每一个细节都要掌握清楚,不能模棱两可,指导我一步步从对科研一无所知到可以独当一面。翁老师还经常指导我和一些实验组合作,在这个过程中,我发现翁老师不光有丰富的理论知识,对实验的细节也驾轻就熟。老师告诉我这些知识就是要靠不断的积累、学习。
% 在这些合作中,我不仅向实验组的老师和同学们学到很多知识,也让我意识到不光要会计算,还要能理解实验的可行性,这对于我今后寻找有价值的课题非常重要。在组里的这几年,翁老师还为我提供了很多学习的机会,包括组里举办的一些workshop,其他单位组织的关于拓扑理论和第一性原理相关的会议。每次会议我都能学习到很多知识,还能有机会见识自己所在的领域里的各位前辈们的科研品味、态度和精神,有时还有机会能结交到一些很优秀的同龄人,这些都是我今后科研道路上的宝贵财富。

% 我还要感谢‪Oleg Yazyev‬老师和吴泉生师兄、张胜男师姐对我科研工作和生活初期的耐心指导。Oleg老师是一个非常耐心的老师,经常对当时几乎什么都不懂的我给予鼓励。他看待问题总是可以做到一针见血,每次组会上,当我刚刚把自己遇到的困难说出来,他就能立马抓住问题的本质,告诉我可以在哪些方面寻求突破。吴师兄能力非常强,不仅有扎实的理论功底上,超强的编程能力,还有非常强的学习能力,独特的科研品味。我在第一性原理计算方面的相关知识几乎是吴师兄手把手教的,师兄总是很有耐心的一边指导我,一边鼓励我,还“骗”我他读研时也像我一样什么都不懂。在师兄的“欺骗”下,我一直自我感觉良好,虽然遇到一些困难,但也能够乐观面对,这使得我在后知后觉中一步步走上科研的正轨。张胜男师姐在生活方面给了我非常非常多的帮助,在科研上我遇到困难时也经常鼓励我,在此表示真心的感谢。

% 我还想感谢方辰老师。和方老师的合作并不多,但是方辰却一直是我的精神偶像。听过方老师的很多报告,也拜读过方老师的很多工作,可以说我在拓扑理论方面的大部分知识是从方老师的工作中学习到的。我的LaSbTe这个工作也是完全依赖于方辰老师层构造的方法。那一次方老师一遍遍认真地帮我修改论文,还告诫我论文里一定不能有低级错误。后来又有幸经常参加方老师组内聚会,聆听老师对科研的认识和见解。在我博后申请方面,方老师也帮了我很大的忙,不仅帮我写推荐信,还认真地教我如何正确使用一些语法。这次的三月会议,也有幸聆听了方老师对我讲解的工作提供的指导意见。唯一遗憾的是我一直因为比较怂,有问题却不敢向方老师请教,总怕被老师嫌弃而丧失了很多次和老师学习的机会。虽然在紧张的科研工作和生活中与方老师的交集不多,但深深被方老师的为人所折服。希望今后也能像方老师一样在科研和生活两方面做好自己。

% 我还想感谢王志俊师兄对我科研工作的指导。王师兄工作非常努力,认真。师兄在科研工作中事无巨细,让我学到了很多。师兄的想法很多,给了我很多材料去探索,我能够顺利毕业也离不开王师兄的指导。

% 我还想感谢每一位合作者,感谢方忠老师,物理所石友国老师,周兴江老师,刘国东老师,钱天老师,王建涛老师,靳常青老师,刘淼老师,雒建林老师,毛寒青老师,遵义师范大学的谭志云老师,人大的雷和畅老师,高嘉成师弟,聂思敏师兄,张坦师兄,崔志海师兄,张薇师姐,宋志达师兄,蒋毅师弟,杨萌师姐,伊长江师兄,孔令元师兄,王阳师姐,吴德胜同学,卜坤师弟,赵建发师兄,樊文辉同学,李勇师弟,刘清波同学等,与你们的合作让我受益匪浅!

% 感谢理论室的王磊老师,孟子杨老师,万源老师,周毅老师,徐力方老师,齐建为老师,于艳梅老师,刘伍明老师,李晶晶老师,边智聪老师,王静静老师等各位老师在科研工作和生活中的帮助!感谢组里徐刚,赵建洲,程秋波,邵德喜,杨健,郭照
% \hbox{\lower-0.7ex\hbox{\scalebox{1}[0.9]{艹}}\lower.1ex\hbox{\kern-1em \scalebox{1}[0.7]{凡}}}
% ,许秋楠,顾越强,张田田,周丽琴,皮涵琦,彭炳睿,李楚豪,李烁辉,刘毓智,潘高培,岳长明,徐远锋,任宏斌,彭士宇,孙松,李哲,陈闯,廖元达,任杰,刘子宏,孙光宇,邵岳林,姬学聪,张中义,张帅,王薇,王瑶,张锴,贾玉锦,吴东宇,邓俊泽,朱天念,高恒,梁英宗,王珊珊,钱晨等师兄弟姐妹们,与你们学习生活,让我成长到了很多!诚然,成年人的世界里没有永远的陪伴,纵有千般不舍,终究还是要前行。愿T03组永远和谐温暖,朝气蓬勃,加油!

% 感谢我的爱人胡志华这些年来对我科研工作一如既往的支持,对我的生活无微不至的关怀。最后我想感谢我的父母在家境拮据的情况下,一直供我读书,支持我读博,甚至出国深造。仅以此论文献给我亲爱的父母亲!

% \cleardoublepage[plain]% 让文档总是结束于偶数页,可根据需要设定页眉页脚样式,如 [noheaderstyle]
% %---------------------------------------------------------------------------%
