\chapter{总结与展望}\label{chap:summary}
本论文主要介绍了我在博士期间的四个研究工作。我的工作主要是基于拓扑能带理论,利用第一性原理,结合Wannier紧束缚模型和$\mathbf{k\cdot p}$模型,研究材料的拓扑性质。我的工作可以分为两个方面:一方面是新的拓扑材料的计算和预言,另一方面是利用现有的理论知识,对已知拓扑体系的物性进行理论解释,进一步发展新的方法,为寻找新的拓扑材料提供理论依据。

在拓扑材料预言方面,我们预言了在超导体HfRuP家族具有拓扑非平庸的电子结构,其中HfRuP属于第二类外尔半金属,而ZrRuAs,ZrRuP和HfRuAs属于拓扑晶体绝缘体。我们与石友国老师团队合作,成功生长了HfRuP和ZrRuAs单晶样品,以这两个材料为代表,研究了其电阻和磁化率随温度变化的规律,发现两个材料都在超导转变温度后有零电阻和迈斯纳效应,成功证明了HfRuP家族的超导性。与此同时,我们与钱天老师和丁洪老师的团队合作,研究了这两个的角分辨光电子能谱,发现与理论计算吻合的非常好。我们预言这个家族的拓扑性质和本征的超导性质相结合极有可能实现拓扑超导,其中HfRuP可能通过外尔半金属相和超导性结合实现拓扑超导,ZrRuAs、ZrRuP或HfRuAs可能通过拓扑晶体绝缘体相和超导性结合实现镜面拓扑超导。

关于LaSbTe的研究和S$_4$外尔半金属的研究构成本论文的第二个方面,这些工作的意义主要体现在进一步深入理解材料拓扑物性,为寻找新的拓扑材料提供理论依据。首先总结一下关于LaSbTe的研究。我们知道所有的晶体点群对称性保护的拓扑态都可以通过堆叠低维的拓扑态来实现。利用实空间构造的方法,方辰老师团队完成了所有磁群从对称性指标到拓扑不变量的映射关系,在拓扑分类方面取得了非常重要的成果。但是这个方法的不足之处是物理图像并不清晰,具体来讲就是在实际材料中,即使知道层构造的方式,但很难理解实空间相应的密勒面可以看作是低维的拓扑态。我们利用层状材料LaSbTe,可以完美的解释这一物理图像。我们发现,这个材料有两个拓扑相,一个是四方的弱拓扑绝缘体相\ti,一个是正交的拓扑晶体绝缘体相\tci~。通过声子谱计算分析,我们知道每一个正交相结构的原胞都可以看作是两个四方相结构的发生结构相变之后沿着$c$轴方向堆叠得到的。利用层构造方法分析后,我们发现四方相\ti~每一层都可以看作是一个二维拓扑绝缘体。正交相\tci~恰好可以看作是位于密勒面(001;$\frac{1}{4}$)和(001;$\frac{3}{4}$)位置的两个\ti~发生形变后沿着c方向的堆叠得到。并且通过这种层构造的方式,可以很容易推导得到的相应的对称性指标和拓扑不变量。拓扑不变量告诉我们这个拓扑晶体拓扑绝缘体\tci~具有沙漏型表面态和($d-2$)维的棱态。我们使用第一性原理结合Wannier函数,Wilson-loop谱的方法,验证了该体系确实存在上述性质。这是第一次找到一个实验可以得到的材料能够生动形象地将实空间的层构造和动量空间的能带拓扑相对应。相信这个例子为我们理解拓扑晶体绝缘体和拓扑绝缘体的关系和物理本质提供了非常好的平台,由此也可以思考通过层构造的方法,设计新的拓扑材料。

由于外尔半金属除了平移对称性之外,不受任何晶体对称性的保护,所以判断外尔半金属没有相应的对称性指标。但是我们发现有一类具有旋转对称性C$_4$和空间反演对称性$I$的组合对称性S$_4$的体系,我们可以通过定义的S$_4$的对称性指标$\eta$和时间反演对称性指标$z_2$的不匹配,即$\eta \neq z_2$来判断体系存在外尔点。并且通过仔细研究一系列锡矿结构的铜基硫族元素化物,发现之前被认为是拓扑绝缘体的一些材料满足$\eta \neq z_2$,因而是外尔半金属。通过Wilson-loop计算陈数,确实证实了在这些材料中存在外尔点,验证了我们的理论。后来,我们还进一步对所有的具有S$_4$对称性的体系进行了系统的研究,定义统一的S$_4$不变量,由此对材料数据库进行了高通量的计算,发现了很多新奇的外尔半金属~\citep{tobedone2019}。相信我们的理论可以在今后的科研中对于理解量子材料的拓扑物性,预言新的拓扑材料方面起到重要的作用~\citep{gapless2020}。
